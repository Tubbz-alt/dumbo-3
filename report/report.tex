\documentclass[12pt,twocolumn]{article}
\usepackage[a4paper, margin=1.5cm, bottom=3cm]{geometry}
\usepackage[utf8]{inputenc}
\usepackage[T1]{fontenc}
\usepackage{hyperref}
\usepackage{url}
\usepackage[english,frenchb]{babel}
\usepackage{csquotes}
\usepackage{amsthm}
\newtheorem{prop}{Propriété}
\newtheorem{dfn}{Définition}
\usepackage{amssymb}
\usepackage{amsmath}
\usepackage{cancel}
\newcommand{\bigO}{\mathcal{O}}
\newcommand{\esN}{\mathbb{N}}
\newcommand{\esR}{\mathbb{R}}
\newcommand{\reg}{\mathcal{R}}
\newcommand{\es}{\emptyset}
\newcommand{\inc}{\subseteq}
\newcommand{\sm}{\setminus}
\usepackage{pgf,tikz}
\usetikzlibrary{arrows}
\usepackage[french,onelanguage]{algorithm2e}
\title{Devoir de Compilation -- Dumbo, un moteur de templates}
\author{Jérémy \textsc{Gheysen}, Guillaume \textsc{Huysmans}}
\date{23 mai 2016}
\hypersetup{
	pdftitle={Devoir de Compilation -- Dumbo, un moteur de templates},
	pdfauthor={Jérémy \textsc{Gheysen}, Guillaume \textsc{Huysmans}},
	pdfsubject={interpreter, templating, engine},
	pdfkeywords={yacc, lex, Python}}
%\bibliographystyle{plain}
\begin{document}
\maketitle


\section{Énoncé}
Il nous est demandé d'implémenter un DSL\footnote{Domain-Specific Language}
afin de gérer la mise en page de données dans un fichier texte (HTML ou non)
à l'aide de PLY, un outil libre dédié à la création de compilateurs.
Ce langage doit implémenter :
\begin{itemize}
\item les opérations arithmétiques et booléennes
\item les concaténations de listes
\item les affectations de variables
\item les boucles sur des listes de chaînes de caractères constantes
\item les conditions
\end{itemize}


\section{Grammaire}
Pour ne pas confondre expressions et instructions, le symbole de variable
\texttt{expr} de l'énoncé a été renommé en \texttt{stmt}.
La structure générale d'un programme est la suivante :
\begin{verbatim}
program : TEXT
        | block
        | TEXT program
        | block program
block   : BEGIN stmt_l END
stmt_l  : stmt SEMICOLON
        | stmt SEMICOLON stmt_l
\end{verbatim}

Les expressions sont décrites
sous leur forme ambiguë pour plus de simplicité
(cf. table \ref{tab:prio}) :
\begin{verbatim}
expr    : INT
        | STRING
        | TRUE
        | FALSE
        | ID
        | LPAREN expr RPAREN
        | NOT expr
        | MINUS expr %prec UNEG
        | LENGTHOF expr
        | LENGTHOF str_l
        | expr PLUS expr
        | expr MINUS expr
        | expr TIMES expr
        | expr DIV expr
        | expr CONCAT expr
        | expr LT expr
        | expr LE expr
        | expr GT expr
        | expr GE expr
        | expr EQUALS expr
        | expr DIFFERENT expr
        | expr AND expr
        | expr OR expr
        | expr XOR expr
\end{verbatim}

\begin{table}[b]
\center
\begin{tabular}{l|l}
	N & \texttt{EQUALS}, \texttt{DIFFERENT} \\
	L & \texttt{COMMA} \\
	L & \texttt{CONCAT} \\
	N & \texttt{OR}, \texttt{XOR} \\
	L & \texttt{AND} \\
	R & \texttt{NOT} \\
	N & \texttt{LT}, \texttt{GT}, \texttt{LE}, \texttt{GE} \\
	N & \texttt{PLUS}, \texttt{MINUS} \\
	N & \texttt{TIMES}, \texttt{DIV} \\
	R & \texttt{UNEG} \\
	N & \texttt{LENGTHOF}
\end{tabular}
\caption{Priorité croissante des opérateurs}
\label{tab:prio}
\end{table}

Les instructions sont les suivantes :
\begin{verbatim}
stmt : PRINT expr
     | PRINT str_l
     | FOR ID IN str_l DO stmt_l ENDFOR
     | FOR ID IN ID DO stmt_l ENDFOR
     | IF expr DO stmt_l ENDIF
     | IF expr DO stmt_l ELSE stmt_l ENDIF
     | ID ASSIGN expr
     | ID ASSIGN str_l
\end{verbatim}

Les listes de chaînes de caractères sont définies de manière à ne pas permettre
de confusion entre une liste d'un élément et une expression entre parenthèses :
\begin{verbatim}
str_l : LPAREN RPAREN
      | LPAREN STRING COMMA RPAREN
      | LPAREN STRING COMMA strs RPAREN
strs  : STRING
      | STRING COMMA strs
\end{verbatim}


\section{Implémentation}
\subsection{Fichier de données}
Le fichier de données est interprété normalement sauf qu'il ne pourra jamais
afficher quoi que ce soit : un \textit{mock} de \texttt{stdout} remplace son
flux de sortie.


\subsection{Listes}
Les listes sont de simples listes chaînées. Cela évite de les construire
naïvement en $\bigO(n^2)$ par \texttt{append} successifs et le fonctionnement
des boucles \texttt{for} reste simple. La fonction associée à ces règles
ne retourne pas de fonction mais directement une instance de \texttt{Node}.


\subsection{if}
On ne peut pas exécuter une instruction directement lorsqu'elle est reconnue :
un \texttt{if} n'aurait aucune utilité si avant d'être détecté, tout son contenu
était de toute façon exécuté, peut-être même avant la création des variables
nécessaires à son bon fonctionnement.

Comme la plupart des autres règles, celles associées aux
\texttt{IF DO (ELSE) ENDIF} ne retournent pas une valeur mais une fonction
qui sera exécutée plus tard au bon moment : celle-ci évalue la condition,
vérifie que c'est bien un booléen puis exécute son <<~contenu positif~>>
si elle est vérifiée. Sinon, si un \texttt{ELSE} existe, l'autre partie est
interprétée récursivement.


\subsection{for}
Comme déjà explicité précédemment, cette règle, comme beaucoup d'autres,
retourne une fonction qui sera exécutée en temps voulu. La principale
caractéristique du \texttt{for} réside dans le parcours de la liste de
\texttt{STRING}s qui lui est spécifiée, soit directement, soit par
l'intermédiaire d'un identificateur. On peut donc alors avoir deux formes de
boucle \texttt{for} : \texttt{FOR IDENTIFIER IN str\_l DO stmt\_l} et
\texttt{FOR IDENTIFIER IN IDENTIFIER DO stmt\_l}.

Pour les deux formes de \texttt{for}, un parcours de la liste donnée et stockée
sous la forme de liste chaînée a été effectué. Une attention particulière a
également été mise sur la portée des variables utilisées, tout particulièrement
le premier \texttt{IDENTIFIER} qui se doit de retrouver sa valeur initiale
(avant l'entrée dans la boucle) dès qu'il sort du bloc de la boucle. 

\subsection{lengthof}
Nous avons décidé d'ajouter l'opérateur \texttt{lengthof} afin de faciliter le
parcours des listes en permettant de connaître leur taille. Celui-ci peut être
utilisé de la manière suivante :
\texttt{LENGTHOF ('a', 'b')} ou \texttt{LENGTHOF variable} si celle-ci
référence une liste. Ceci est très pratique dans de multiples cas de figure.
Mais ce n'est pas tout, il peut également être utilisé
sur des \textit{strings} :
le langage est ainsi plus expressif puisqu'il était jusqu'alors incapable
de calculer la longueur de chaînes de caractères.

\section{Problèmes rencontrés}
L'objet fourni par PLY en paramètre de chaque fonction
associée à une règle est toujours le même. On ne peut ainsi
pas espérer accéder au même \texttt{p[1]} depuis une fonction
que l'on retourne, il est nécessaire de créer une \textit{closure}
avec des variables contenant les éléments <<~réduits~>>.
Si on ne le fait pas, l'interpréteur peut partir en récursion infinie
ou calculer n'importe quoi alors que le programme semble cohérent.

La vérification statique des types n'est pas simple puisqu'à chaque bloc,
il faudrait associer un environnement. Ici, le \texttt{parser} retourne
directement une fonction qui peut être appelée pour exécuter le script
empêchant ainsi ce type de vérification.
Python les contrôle aussi à l'exécution.


\section{Conclusion}
TODO


%\bibliography{comp}
\end{document}
